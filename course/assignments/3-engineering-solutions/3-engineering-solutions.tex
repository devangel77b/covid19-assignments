\documentclass{exam}

\newcommand{\myroot}{../../..}
\usepackage[hw]{\myroot/course}

\title{COVID-19 Assignment \#3: Engineering solutions}
\author{D Evangelista}
\date{\today}
\duedate{\printdate{3/31/2020}}

\begin{document}
\maketitle

\begin{abstract}
Did you know there are some members of the USNA community who need hand sanitizer, gloves, masks because they're caring for loved ones with special needs? However, these items have spotty availability due to panic buying and rampant price gouging: an engineering problem that doesn't have a high tech solution. You can apply the engineering design process you have been learning about to the problem at hand in several ways. 
\end{abstract}

\begin{questions}
\question Apply engineering design process to setting up a collection point for supplies (soap, wipes, hand sanitizer, gloves, masks) anyone wishes to donate for use by members of the USNA community who need them and cannot get them, clean/sterile/unopened. 

\question Design and implement a process to make improvised hand sanitizer from $>70\%$ isopropanol and aloe gel. You may wish to consider sources for isopropanol, gel, and suitable pump containers; quality control and infection control steps; and how to effectively distribute your product and instruct consumers as to its proper use. 

\question To allow people to economize use of actual surgical masks, there are a number of patterns for masks online; the main use would be to prevent casual face touches and to make the actual masks more available for those who really need them. Design the masks and implement a manufacturing process to source adequate numbers for a target market of your choosing. 

\question Effective ``messaging'' is also a product. Design and implement a poster, video, or social media campaign to encourage the community to adopt infection control measures. You should consider metrics suitable for measuring your campaigns effectiveness. 
\end{questions}
\end{document}
